\documentclass[a4paper,10pt,ngerman]{scrartcl}
\usepackage{babel}
\usepackage[T1]{fontenc}
\usepackage[utf8x]{inputenc}
\usepackage[a4paper,margin=2.5cm,footskip=0.5cm]{geometry}

% Die nächsten vier Felder bitte anpassen:
\newcommand{\Aufgabe}{Aufgabe 4: Nandu} % Aufgabennummer und Aufgabennamen angeben
\newcommand{\TeamId}{00988}                       % Team-ID aus dem PMS angeben
\newcommand{\TeamName}{BitShifter}                 % Team-Namen angeben
\newcommand{\Namen}{Filip Zelinskyi}           % Namen der Bearbeiter/-innen dieser Aufgabe angeben
 
% Kopf- und Fußzeilen
\usepackage{scrlayer-scrpage, lastpage}
\setkomafont{pageheadfoot}{\large\textrm}
\lohead{\Aufgabe}
\rohead{Team-ID: \TeamId}
\cfoot*{\thepage{}/\pageref{LastPage}}

% Position des Titels
\usepackage{titling}
\setlength{\droptitle}{-1.0cm}

% Für mathematische Befehle und Symbole
\usepackage{amsmath}
\usepackage{amssymb}

% Für Bilder
\usepackage{graphicx}

% Für Algorithmen
\usepackage{algorithm}
\usepackage{algpseudocode}
\algnewcommand{\algorithmicand}{\textbf{ and }}
\algnewcommand{\algorithmicor}{\textbf{ or }}
\algnewcommand{\OR}{\algorithmicor}
\algnewcommand{\AND}{\algorithmicand}
\algnewcommand{\var}{\texttt}
\algnewcommand{\TRUE}{\textbf{true}}
\algnewcommand{\FALSE}{\textbf{false}}
\renewcommand{\algorithmiccomment}[1]{\hfill$\triangleright$\textit{#1}}

% Für Quelltext
\usepackage{listings}
\usepackage{color}
\definecolor{mygreen}{rgb}{0,0.6,0}
\definecolor{mygray}{rgb}{0.5,0.5,0.5}
\definecolor{mymauve}{rgb}{0.58,0,0.82}
\lstset{
  keywordstyle=\color{blue},commentstyle=\color{mygreen},
  stringstyle=\color{mymauve},rulecolor=\color{black},
  basicstyle=\footnotesize\ttfamily,numberstyle=\tiny\color{mygray},
  captionpos=b, % sets the caption-position to bottom
  keepspaces=true, % keeps spaces in text
  numbers=left, numbersep=5pt, showspaces=false,showstringspaces=true,
  showtabs=false, stepnumber=2, tabsize=2, title=\lstname
}
\lstdefinelanguage{JavaScript}{ % JavaScript ist als einzige Sprache noch nicht vordefiniert
  keywords={break, case, catch, continue, debugger, default, delete, do, else, finally, for, function, if, in, instanceof, new, return, switch, this, throw, try, typeof, var, void, while, with},
  morecomment=[l]{//},
  morecomment=[s]{/*}{*/},
  morestring=[b]',
  morestring=[b]",
  sensitive=true
}

% Diese beiden Pakete müssen zuletzt geladen werden
%\usepackage{hyperref} % Anklickbare Links im Dokument
\usepackage{cleveref}

% Daten für die Titelseite
\title{\textbf{\Huge\Aufgabe}}
\author{\LARGE Team-ID: \LARGE \TeamId \\\\
	    \LARGE Team-Name: \LARGE \TeamName \\\\
	    \LARGE Bearbeiter dieser Aufgabe: \\ 
	    \LARGE \Namen\\\\}
\date{\LARGE\today}

\begin{document}

\maketitle
\tableofcontents

\vspace{0.5cm}

\section{Abstract}

Die Lösung beruht auf der Simulation von Logik-Gattern. Die Suche nach der Antwort beginnt nicht am Anfang bei den Lichtquellen, sondern am Ende bei den Lichtempfängern.

In diesem Dokument wird der Lösungsansatz beschrieben, die Umsetzung auf JavaScript erläutert, die Lösungen für die Beispiele und die interaktive Web-Anwendung als Zusatzaufgabe vorgestellt.

\section{Lösungsidee}

Jeder Baustein kann als ein Logikgatter dargestellt werden. Der blaue Baustein: $\lnot (a \land b)$, der rote: $\lnot a$ und der blaue einfach als ein Draht, der das Signal nicht beeinflusst.
Um das Ergebnis am Empfänger \(L_i\) für den aktuellen Zustand der Lichtquellen zu bestimmen, muss festgestellt werden, von welchen Logikgattern und welchen Lichtquellen die Antwort beeinflusst wird. Dadurch kann eine Funktion erstellt werden, die bei Eingabe des Zustands relevanter Quellen sofort die Antwort liefert. Dies wird durch das Durchlaufen der oberhalb liegenden Bausteine vom Lichtempfänger aus erreicht, bis der Zeiger auf Vakuum oder eine Lichtquelle stößt.
Eine solche Funktion wird für alle Empfänger auf der Karte erstellt. Die Ausgabe wird für alle möglichen Kombinationen der Lichtquellen (\(2^{|Q|}\)) berechnet.

\newpage
\section{Umsetzung}

Zunächst erfolgt die Initialisierung globaler Variablen zur Verwaltung von Lichtquellen und Lichtempfängern. In diesem Abschnitt werden auch die Funktionen für die Simulation der Logikgatter definiert.

\begin{algorithm}
    \caption{Initialisierung globaler Variablen und Funktionen für die Simulation von Logikgattern}
\begin{algorithmic}
\Require{input[ROW][COL]} \Comment{Größe des Feldes aus der Eingabe}
\State $QuelleStatus \gets HashMap(name, bool)$
\State $Empfaenger \gets HashMap(name, SolveFunction)$

\Function{WeissBaustein}{a: bool, b: bool}
\State \Return $\lnot (a \land b)$
\EndFunction

\Function{RotBaustein}{a: bool}
\State \Return $\lnot a$
\EndFunction
\end{algorithmic}
\end{algorithm}

Da es vorkommen kann, dass für einen weißen Baustein beide vorherigen Elemente von Bedeutung sind, benötigen wir 2 x-Werte, an denen sich der Baustein $matrix[row][col]$ befindet. Gleiches gilt, wenn beispielsweise bei einem roten Baustein die Lichtquelle vom Sensor gefunden werden muss. Hier wird ein Zähler $blockIndex$ initialisiert, der modulo 2 berechnet wird, exklusive Vakuum.

\begin{algorithm}
    \caption{Suche den x-Wert des geeigneten zweiten Baustein-Teils}
    \begin{algorithmic}
        \Function{GetBlockPair}{matrix[ROW][COL], col, row}
        \State $blockIndex \gets 0$
        \For{$x \gets 0$ to $col$}
        \If{$matrix[row][x] = X$}
        \State $blockIndex \gets 0$
        \Else
        \State $blockIndex \gets (blockIndex + 1) \mod 2$
        \EndIf
        \EndFor
        \If{$blockIndex > 0$}
        \State \Return $col - 1$
        \Else
        \State \Return $col + 1$
        \EndIf
        \EndFunction
    \end{algorithmic}
\end{algorithm}

Die Lösungsfunktion sollte rekursiv bis zur Lichtquelle vordringen und eine Sammlung von relevanten Logikgattern, die die Antwort beeinflussen, in Form einer Funktion ausgeben.

\begin{algorithmic}
        \Function{Solve}{matrix[ROW][COL], x, y}
        \If{$x < 0$ \OR $y < 0$ \OR $matrix[y][x] = X$}
        \State \Return \FALSE;
        \EndIf
        \State $cell \gets matrix[y][x]$
        \If{cell beginnt mit $Q$}
        \State \Return $QuelleStatus[cell]$
        \ElsIf{$cell = B$}
        \State \Return $\textbf{Solve}(matrix, x, y - 1)$
        \ElsIf{$cell = R$}
        \State \Return $\textbf{RotBaustein}(\textbf{Solve}(matrix, x, y - 1))$
        \ElsIf{$cell = r$}
        \State \Return $\textbf{Solve}(matrix, \textbf{GetBlockPair}(matrix, x, y), y - 1)$
        \ElsIf{$cell = W$}
        \State \Return $\textbf{WeissBaustein}(\textbf{Solve}(matrix, x, y - 1), \textbf{Solve}(matrix, \textbf{GetBlockPair}(matrix, x, y), y - 1))$
        \EndIf
        \EndFunction
    \end{algorithmic}

\vspace{0.5cm}

\newpage
\section{Beispiele}

\subsection*{Nandu 1}
\begin{lstlisting}[language=C++]
cat examples/nandu1.txt | node nandu.js
Q1      Q2      L1      L2
Aus     Aus     Ein     Ein
Ein     Aus     Ein     Ein
Aus     Ein     Ein     Ein
Ein     Ein     Aus     Aus
\end{lstlisting}

\subsection*{Nandu 2}
\begin{lstlisting}[language=C++]
cat examples/nandu2.txt | node nandu.js
Q1      Q2      L1      L2
Aus     Aus     Aus     Ein
Ein     Aus     Aus     Ein
Aus     Ein     Aus     Ein
Ein     Ein     Ein     Aus
\end{lstlisting}

\subsection*{Nandu 3}
\begin{lstlisting}[language=C++]
cat examples/nandu3.txt | node nandu.js
Q1      Q2      Q3      L1      L2      L3      L4
Aus     Aus     Aus     Ein     Aus     Aus     Ein
Ein     Aus     Aus     Aus     Ein     Aus     Ein
Aus     Ein     Aus     Ein     Aus     Ein     Ein
Ein     Ein     Aus     Aus     Ein     Ein     Ein
Aus     Aus     Ein     Ein     Aus     Aus     Aus
Ein     Aus     Ein     Aus     Ein     Aus     Aus
Aus     Ein     Ein     Ein     Aus     Ein     Aus
Ein     Ein     Ein     Aus     Ein     Ein     Aus
\end{lstlisting}

\subsection*{Nandu 4}
\begin{lstlisting}[language=C++]
cat examples/nandu4.txt | node nandu.js
Q1      Q2      Q3      Q4      L1      L2
Aus     Aus     Aus     Aus     Aus     Aus
Ein     Aus     Aus     Aus     Aus     Aus
Aus     Ein     Aus     Aus     Ein     Aus
Ein     Ein     Aus     Aus     Aus     Aus
Aus     Aus     Ein     Aus     Aus     Ein
Ein     Aus     Ein     Aus     Aus     Ein
Aus     Ein     Ein     Aus     Ein     Ein
Ein     Ein     Ein     Aus     Aus     Ein
Aus     Aus     Aus     Ein     Aus     Aus
Ein     Aus     Aus     Ein     Aus     Aus
Aus     Ein     Aus     Ein     Ein     Aus
Ein     Ein     Aus     Ein     Aus     Aus
Aus     Aus     Ein     Ein     Aus     Aus
Ein     Aus     Ein     Ein     Aus     Aus
Aus     Ein     Ein     Ein     Ein     Aus
Ein     Ein     Ein     Ein     Aus     Aus
\end{lstlisting}

\subsection*{Nandu 5}
\begin{lstlisting}[language=C++]
cat examples/nandu5.txt | node nandu.js
Q1      Q2      Q3      Q4      Q5      Q6      L1      L2      L3      L4      L5
Aus     Aus     Aus     Aus     Aus     Aus     Aus     Aus     Aus     Ein     Aus
Ein     Aus     Aus     Aus     Aus     Aus     Ein     Aus     Aus     Ein     Aus
Aus     Ein     Aus     Aus     Aus     Aus     Aus     Aus     Aus     Ein     Aus
Ein     Ein     Aus     Aus     Aus     Aus     Ein     Aus     Aus     Ein     Aus
Aus     Aus     Ein     Aus     Aus     Aus     Aus     Aus     Aus     Ein     Aus
Ein     Aus     Ein     Aus     Aus     Aus     Ein     Aus     Aus     Ein     Aus
Aus     Ein     Ein     Aus     Aus     Aus     Aus     Aus     Aus     Ein     Aus
Ein     Ein     Ein     Aus     Aus     Aus     Ein     Aus     Aus     Ein     Aus
Aus     Aus     Aus     Ein     Aus     Aus     Aus     Aus     Ein     Aus     Aus
Ein     Aus     Aus     Ein     Aus     Aus     Ein     Aus     Ein     Aus     Aus
Aus     Ein     Aus     Ein     Aus     Aus     Aus     Aus     Ein     Aus     Aus
Ein     Ein     Aus     Ein     Aus     Aus     Ein     Aus     Ein     Aus     Aus
Aus     Aus     Ein     Ein     Aus     Aus     Aus     Aus     Ein     Aus     Aus
Ein     Aus     Ein     Ein     Aus     Aus     Ein     Aus     Ein     Aus     Aus
Aus     Ein     Ein     Ein     Aus     Aus     Aus     Aus     Ein     Aus     Aus
Ein     Ein     Ein     Ein     Aus     Aus     Ein     Aus     Ein     Aus     Aus
Aus     Aus     Aus     Aus     Ein     Aus     Aus     Aus     Aus     Ein     Ein
Ein     Aus     Aus     Aus     Ein     Aus     Ein     Aus     Aus     Ein     Ein
Aus     Ein     Aus     Aus     Ein     Aus     Aus     Aus     Aus     Ein     Ein
Ein     Ein     Aus     Aus     Ein     Aus     Ein     Aus     Aus     Ein     Ein
Aus     Aus     Ein     Aus     Ein     Aus     Aus     Aus     Aus     Ein     Ein
Ein     Aus     Ein     Aus     Ein     Aus     Ein     Aus     Aus     Ein     Ein
Aus     Ein     Ein     Aus     Ein     Aus     Aus     Aus     Aus     Ein     Ein
Ein     Ein     Ein     Aus     Ein     Aus     Ein     Aus     Aus     Ein     Ein
Aus     Aus     Aus     Ein     Ein     Aus     Aus     Aus     Aus     Ein     Ein
Ein     Aus     Aus     Ein     Ein     Aus     Ein     Aus     Aus     Ein     Ein
Aus     Ein     Aus     Ein     Ein     Aus     Aus     Aus     Aus     Ein     Ein
Ein     Ein     Aus     Ein     Ein     Aus     Ein     Aus     Aus     Ein     Ein
Aus     Aus     Ein     Ein     Ein     Aus     Aus     Aus     Aus     Ein     Ein
Ein     Aus     Ein     Ein     Ein     Aus     Ein     Aus     Aus     Ein     Ein
Aus     Ein     Ein     Ein     Ein     Aus     Aus     Aus     Aus     Ein     Ein
Ein     Ein     Ein     Ein     Ein     Aus     Ein     Aus     Aus     Ein     Ein
Aus     Aus     Aus     Aus     Aus     Ein     Aus     Aus     Aus     Ein     Aus
Ein     Aus     Aus     Aus     Aus     Ein     Ein     Aus     Aus     Ein     Aus
Aus     Ein     Aus     Aus     Aus     Ein     Aus     Aus     Aus     Ein     Aus
Ein     Ein     Aus     Aus     Aus     Ein     Ein     Aus     Aus     Ein     Aus
Aus     Aus     Ein     Aus     Aus     Ein     Aus     Aus     Aus     Ein     Aus
Ein     Aus     Ein     Aus     Aus     Ein     Ein     Aus     Aus     Ein     Aus
Aus     Ein     Ein     Aus     Aus     Ein     Aus     Aus     Aus     Ein     Aus
Ein     Ein     Ein     Aus     Aus     Ein     Ein     Aus     Aus     Ein     Aus
Aus     Aus     Aus     Ein     Aus     Ein     Aus     Aus     Ein     Aus     Aus
Ein     Aus     Aus     Ein     Aus     Ein     Ein     Aus     Ein     Aus     Aus
Aus     Ein     Aus     Ein     Aus     Ein     Aus     Aus     Ein     Aus     Aus
Ein     Ein     Aus     Ein     Aus     Ein     Ein     Aus     Ein     Aus     Aus
Aus     Aus     Ein     Ein     Aus     Ein     Aus     Aus     Ein     Aus     Aus
Ein     Aus     Ein     Ein     Aus     Ein     Ein     Aus     Ein     Aus     Aus
Aus     Ein     Ein     Ein     Aus     Ein     Aus     Aus     Ein     Aus     Aus
Ein     Ein     Ein     Ein     Aus     Ein     Ein     Aus     Ein     Aus     Aus
Aus     Aus     Aus     Aus     Ein     Ein     Aus     Aus     Aus     Ein     Ein
Ein     Aus     Aus     Aus     Ein     Ein     Ein     Aus     Aus     Ein     Ein
Aus     Ein     Aus     Aus     Ein     Ein     Aus     Aus     Aus     Ein     Ein
Ein     Ein     Aus     Aus     Ein     Ein     Ein     Aus     Aus     Ein     Ein
Aus     Aus     Ein     Aus     Ein     Ein     Aus     Aus     Aus     Ein     Ein
Ein     Aus     Ein     Aus     Ein     Ein     Ein     Aus     Aus     Ein     Ein
Aus     Ein     Ein     Aus     Ein     Ein     Aus     Aus     Aus     Ein     Ein
Ein     Ein     Ein     Aus     Ein     Ein     Ein     Aus     Aus     Ein     Ein
Aus     Aus     Aus     Ein     Ein     Ein     Aus     Aus     Aus     Ein     Ein
Ein     Aus     Aus     Ein     Ein     Ein     Ein     Aus     Aus     Ein     Ein
Aus     Ein     Aus     Ein     Ein     Ein     Aus     Aus     Aus     Ein     Ein
Ein     Ein     Aus     Ein     Ein     Ein     Ein     Aus     Aus     Ein     Ein
Aus     Aus     Ein     Ein     Ein     Ein     Aus     Aus     Aus     Ein     Ein
Ein     Aus     Ein     Ein     Ein     Ein     Ein     Aus     Aus     Ein     Ein
Aus     Ein     Ein     Ein     Ein     Ein     Aus     Aus     Aus     Ein     Ein
Ein     Ein     Ein     Ein     Ein     Ein     Ein     Aus     Aus     Ein     Ein
\end{lstlisting}
\pagebreak

\section{Quellcode}
\begin{lstlisting}[language=JavaScript]
const w = (a, b) => !(a && b);
const r = (a) => !a;

const QuelleStatus = {};
const Empfaenger = {}

const getBlockPair = (matrix, col, row) => {
  let blockIndex = 0;
  for (let x = 0; x < col; x++) {
    if (matrix[row][x] === 'X') blockIndex = 0;
    else blockIndex = (blockIndex + 1) % 2;
  }
  return blockIndex ? col - 1 : col + 1;
}

const solve = (matrix, x, y) => {
  if (x < 0 || y < 0 || matrix[y][x] === 'X') return false;
  const cell = matrix[y][x];
  if (cell.startsWith("Q")) return QuelleStatus[cell]
  const Actions = {
    B: () => solve(matrix, x, y - 1),
    R: () => r(solve(matrix, x, y - 1)),
    W: () => w(solve(matrix, x, y - 1), solve(matrix, getBlockPair(matrix, x, y), y - 1)),
    r: () => solve(matrix, getBlockPair(matrix, x, y), y),
  }
  return Actions[cell]();
}

process.stdin.on('data', async data => {
  // Bearbeitung der Eingabe wurde gekuerzt
  // Es wurde die Matrix input[y][x] erstellt

  for (let y = row - 1; y >= 0; y--) {
    for (let x = 0; x < col; x++) {
      const cell = input[y][x];
      if (cell.startsWith("L")) {
        Empfaenger[cell] = () => solve(input, x, y - 1);
      } else if (cell.startsWith("Q")) {
        QuelleStatus[cell] = false;
      }
    }
  }

  // Gekuerzt: Ausgabe fuer alle Kombinationen
})

\end{lstlisting}

\section{Zusatzaufgabe}

Meine Webanwendung für die interaktive Visualisierung von Bausteinen wurde komplett in reinem JavaScript, ohne die Verwendung externer Bibliotheken, entwickelt. Dies gewährleistet maximale Flexibilität und Kontrolle über alle Funktionen des Projekts, wobei der Fokus auf der Canvas-Technologie liegt. Die Inspiration für das Design stammt von ähnlichen Webanwendungen wie Figma und GeoGebra.

Im Hintergrund ist ein Koordinatensystem sichtbar, um das sich die Bausteine magnetisch bewegen können. Das Originaldesign des Aufgabenblatts wurde für die Darstellung der Bausteine verwendet.
Sie können die Datei index.html problemlos öffnen, um sich das Projekt anzusehen.

Hier sind die wichtigsten Funktionen:

\subsubsection*{Pan \& Zoom}
\begin{itemize}
    \item Leichte Navigation durch Maus- und Tastatursteuerung.
    \item Funktioniert sowohl auf Tracking-Pads mit Gestenerkennung, ähnlich wie in Figma, als auch mit einer herkömmlichen Maus.
    \item Shortcuts ermöglichen den Wechsel zwischen Instrumenten:
    \begin{itemize}
        \item \texttt{M/H} - Bewegen der Ansicht
        \item \texttt{V} - Auswahl von Bausteinen
        \item \texttt{Strg + Plus} - Hereinzoomen
        \item \texttt{Strg + Minus} - Herauszoomen
    \end{itemize}
\end{itemize}

\subsubsection*{Bewegen von Bausteinen}
\begin{itemize}
    \item Mit dem Auswahlwerkzeug (Taste \texttt{V}) können Bausteine auf dem Feld verschoben werden.
    \item Möglichkeit, mehrere Elemente gleichzeitig auszuwählen und zu bearbeiten, indem ein Rechteck gezogen wird, das die gewünschten Elemente berührt.
\end{itemize}

\subsubsection*{Rotieren von roten Bausteinen}
\begin{itemize}
    \item Rote Bausteine können durch doppeltes Klicken verschoben werden.
\end{itemize}

\subsubsection*{Ein- und Ausschalten von Lichtquellen}
\begin{itemize}
    \item Taschenlampe kann durch doppeltes Klicken ein- oder ausgeschaltet werden.
\end{itemize}

\subsubsection*{Löschen von Elementen}
\begin{itemize}
    \item Ausgewählte Elemente können durch Drücken der \texttt{Backspace}-Taste gelöscht werden.
\end{itemize}

\subsubsection*{Importieren von Beispielaufgaben}
\begin{itemize}
    \item Textdateien im Format von Beispieldateien können durch Ziehen der Datei auf die Webseite importiert werden.
\end{itemize}

\subsubsection*{Einfügen von Elementen}
\begin{itemize}
    \item Ein Element kann durch Klicken an einer beliebigen Stelle auf dem Koordinatenfeld eingefügt werden. Im daraufhin geöffneten Kontextmenü kann ein Element ausgewählt werden.
\end{itemize}

\begin{figure}[!ht]
    \centering
    \includegraphics[width=0.5\linewidth]{Nandu5.png}
    \caption{Bildschirmaufnahme der Webseite, auf der Beispiel 5 dargestellt ist.}
    \label{fig:nandu5}
\end{figure}

\end{document}
